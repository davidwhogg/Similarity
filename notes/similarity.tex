\documentclass[12pt,letterpaper]{article}

\newcommand{\documentname}{\textsl{Note}}

\begin{document}\sloppy\sloppypar\raggedbottom\frenchspacing

\section*{\raggedright How similar are two noisily observed vectors?}

Imagine that $y_1$ and $y_2$ are two observed data points. Think of
them as $D$-dimensional vectors in a vector space. They could be two
spectra of two stars (Hawkins's problem), or they could be the
detailed element abundances for two stars (Ness's problem). How can we
tell if the two objects observed are intrinsically identical? Would
they appear identical if we had better data?

One option would be to think of making two hypotheses, to wit, ``H1:
The two objects are identical'', and ``H2: The two objects are
different''. Then we could compute the marginalized likelihoods of
these two hypotheses under some priors over the nuisance parameters of
relevance. That's sensible (and more-or-less what we did in S. M. Oh's
paper).

Another option would be to convert this problem into one of parameter
estimation: That is, compute a marginalized likelihood or marginalized
posterior probability density (pdf) for the \emph{difference} between
the two objects, and see if that likelihood or posterior pdf puts
significant density on zero---on a null or vanishing difference
vector.

For various reasons that are (perhaps) badly justified, we are going
to take the second approach in this \documentname...HOGG

\end{document}
