\documentclass[12pt,letterpaper]{article}

% typography
\newcommand{\documentname}{\textsl{Note}}
\newcommand{\equationname}{equation}

% math
\newcommand{\tra}[1]{{#1}^{\mathsf{T}}}
\newcommand{\inv}[1]{{#1}^{-1}}
\renewcommand{\det}[1]{||{#1}||}
\newcommand{\given}{\,|\,}
\newcommand{\dd}{\mathrm{d}}
\newcommand{\normal}{\mathcal{N}}

\begin{document}\sloppy\sloppypar\raggedbottom\frenchspacing

\section*{\raggedright How similar are two noisily observed vectors?}

\noindent
\textbf{David W. Hogg}\\
\textsl{CCPP, NYU Physics \\
        Flatiron Institute \\
        MPIA}

\bigskip

\paragraph{Abstract:}
I present a general (although assumption-laden) method for asking the
question: Are two independently and noisily observed objects in fact
identical? My approach is to turn this question into a parameter
estimation, estimating the (latent) difference between the objects,
and evaluating the posterior density at vanishing difference. The
method accounts both for the fact that the (latent, never
instantiated) noiseless vectors are drawn from a compact pdf, and that
there are differences in the vectors that might not be considered
significant, either because of measurement noise, or because of
intrinsic differences that somehow ``don't matter''. The approach
makes heavy use of Gaussians for their mathematical simplicity. In a
sensible limit, the answer reduces to pure chi-squared difference
between the observations. I discuss the results in the context of
stellar spectroscopy (stellar twins) and chemical-abundance-space
measurements of stars (chemical tagging).

\paragraph{Introduction:}
Imagine that $y_1$ and $y_2$ are two observed data points. Think of
them as $D$-dimensional vectors in a vector space. They could be two
spectra of two stars (Hawkins's problem), or they could be the
detailed element abundances for two stars (Ness's problem). How can we
tell if the two objects observed are intrinsically identical? Would
they appear identical if we had better data?

One option would be to think of making two hypotheses, to wit, ``H1:
The two objects are identical'', and ``H2: The two objects are
different''. Then we could compute the marginalized likelihoods of
these two hypotheses under some priors over the nuisance parameters of
relevance. That's sensible (and more-or-less what we did in S. M. Oh's
paper).

Another option would be to convert this problem into one of parameter
estimation: That is, compute a marginalized likelihood or marginalized
posterior probability density (pdf) for the \emph{difference} between
the two objects, and see if that likelihood or posterior pdf puts
significant density on zero---on a null or vanishing difference
vector.

For various reasons that are (perhaps) badly justified, we are going
to take the second approach in this \documentname. That is, we are
going to estimate the difference between the objects given their
(presumed noisy) measurements, and find out if that difference is
consistent with zero, or with the kind of difference that we would count
as effectively zero (more on this below).

Before we start, one side note: I usually recommend that one just
compute chi-squared between the vectors. This is defined\footnote{In
  \equationname~(\ref{eq:chisquared}), I provide a linear-algebra
  version of the chi-squared formula, which is usually given as a sum
  over independent data points, like
  \[\chi^2\equiv\sum_d\frac{[y_{1d}-y_{2d}]^2}{\sigma_{1d}^2+\sigma_{2d}^2} \quad .\]
  The linear-algebra expression I give in (\ref{eq:chisquared}) is
  identical to this classic sum in the case that the covariance matrix
  $C_1 + C_2$ is diagonal, with variances $\sigma_{1d}^2+\sigma_{2d}^2$ on the
  diagonal.} as
\begin{eqnarray}
\chi^2 &\equiv&
\tra{[y_1 - y_2]}\cdot\inv{[C_1 + C_2]}\cdot [y_1 - y_2] \quad ,
\label{eq:chisquared}
\end{eqnarray}
where, implicitly, $y_1$ and $y_2$ are $D$-dimensional column vectors,
and $C_1$ and $C_2$ are the $D\times D$ covariance matrices that
express our beliefs about the noise variance on the measurements $y_1$
and $y_2$. In this case (that is, if you just compute $\chi^2$) we say
that the two objects are identical if the $\chi^2$ is below some
threshold. That is simple, easy, and probabilistically
justifiable. The objections are manifold. One is that, as the data get
better (the determinants of the $C_1$ and $C_2$ matrices get smaller),
it becomes increasingly harder to conclude similarity or
identicality. To this I say: \emph{Duh!} That will be true of any
correct method. Another objection is that $\chi^2$ treats all elements
of the data (all directions in the vector space) identically, or at
least weights them by their measurement quality rather than their
\emph{usefulness for determining dissimilarity}. If some dimensions
are better than others for determining identicality, $\chi^2$ is blind
to that. That is a significant objection, and the one I
address---below!---in this \documentname.

\paragraph{Assumptions:}
I am going to make the following assumptions, each one of which is
questionable and unpleasant, but each one of which improves our
computational tractability.
\begin{enumerate}\itemsep0ex
\item \emph{Noise:} Observed vectors $y_1$ and $y_2$ are noisy
  measurements of some latent, unobserved \emph{true} vectors $x_1$
  and $x_2$, and the differences between the observed and true vectors
  are drawn from a $D$-dimensional Gaussian distribution with zero
  mean and known variance tensors $C_1$ and $C_2$, respectively. The
  noise draws are independent (but not identically distributed). There
  are no outliers or non-Gaussianities, and the variance tensors are
  not just known but correct. These covariance matrices can contain
  infinite eigenvalues\footnote{HOGG: TBD: Say something about what an eigenvalue is,
    and in the special case of the diagonal matrix.} (that is,
  components or eigen-directions about which there is no information),
  but they probably shouldn't contain zero eigenvalues (that is,
  directions about which there is perfect knowledge).
\item \emph{Prior:} The true vectors are drawn independently from a
  Gaussian distribution of mean $\mu$ and variance tensor $\frac{1}{2}\,V$. This
  variance tensor might have have some zero eigenvalues (that is, it
  might be low-rank), or it might have some infinite eigenvalues; we
  will write down equations that are safe to these possibilities. By
  some magic or algorithm or intuition, the mean vector $\mu$ and
  variance tensor $\frac{1}{2}\,V$ are known in advance.
\item \emph{Intrinsic scatter:} There are certain kinds of differences
  between vectors that are unimportant to us. That is, if they are
  different in these uninteresting ways, we don't consider them to be
  different. We will assume that these intrinsic differences are also
  drawn from a Gaussian, but this time with zero mean, and variance
  tensor $W$, again known in advance. This variance tensor $W$ better
  be compact---in some sense---relative to the prior variance $V$, or
  else all the vectors will be considered similar! Interestingly, the
  variance $W$ can be made compact in the relevant ways by having
  small eigenvectors, or by being low-rank, or both. Even more
  interestingly, it can even have very large (infinite, even)
  eigenvectors provided that it is low rank. More on this below.
  If you want the simple case of just measurement noise and a prior,
  then this variance $W$ can be set to precisely zero in what follows.
\end{enumerate}

Our model is going to be that the difference between the observed
spectra is composed of a true, interesting component, a true,
uninteresting component, and some measurement noise. We are going to
marginalize out everything except the true, interesting
difference. Our marginalized posterior pdf for this interesting
difference will produce the scalar we seek. In equations, our model is:
\begin{eqnarray}
  y_1 - y_2 &=& \Delta + \delta + \epsilon
\\
  p(\Delta) &=& \normal(\Delta\given 0,V)
\\
  p(\delta) &=& \normal(\delta\given 0,W)
\\
  p(\epsilon) &=& \normal(\epsilon\given 0,C_1+C_2)
\quad ,
\end{eqnarray}
where $\Delta$ is the difference that is interesting and important
(and that will be zero if the vectors are truly identical), $\delta$
is the difference that is uninteresting but real, and $\epsilon$ is
pure measurement noise.

\paragraph{Inference:}
From these, we can construct a marginalized posterior pdf
$p(\Delta\given y_1,y_2)$ for the true, interesting difference
$\Delta$, by multiplying together the pdfs and integrating out
$\epsilon$ and $\delta$. This marginalization produces
\begin{eqnarray}
  p(\Delta\given y_1,y_2) &=& \normal(\Delta\given\mu,\Sigma)
\\
  \Sigma &\equiv& \inv{(\inv{[V]} + \inv{[C_1 + C_2 + W]})}
\\
  \mu &\equiv& \Sigma\cdot\inv{[C_1 + C_2 + W]}\cdot[y_1-y_2]
\quad ,
\end{eqnarray}
where $C$ is a variance matrix, and $\mu$ is an expectation value.
These results were found by making heavy use of the following
exact identity\footnote{For which I owe great thank to Boris
Leistedt (NYU).} for Gaussian functions:
\begin{eqnarray}
  \normal(x\given a,A)\,\normal(x\given b,B) &\equiv& \normal(a\given b,A+B)\,\normal(x\given c,C)
\\
  C &\equiv& \inv{[\inv{A} + \inv{B}]}
\\
  c &\equiv& C\cdot [\inv{A}\cdot a + \inv{B}\cdot b]
\quad .
\end{eqnarray}

There are many comments to be made here. One is that the matrix $V$
(which appears in our fomula for $\Sigma$ and therefore also $\mu$)
does not necessarily have an inverse! In which case, inverting it can
be avoided by noting the following identity:
\begin{eqnarray}
  \inv{[\inv{A} + \inv{B}]} &\equiv& A\cdot\inv{[A + B]}\cdot B
\quad .
\end{eqnarray}
That is, we do not require that the prior variance $V$ be positive
definite; it only need be non-negative semi-definite.
Another comment to be made is that as $V$ gets full-rank and large
in all directions, the expectation $\mu$ approaches just the difference
$[y_1-y_2]$.
Another comment is that the variance $W$ appears in parallel with the
noise covariances. This is not surprising, because $W$ is just another
source of noise in this context. It is intrinsic noise. Another way
to put it: Uninteresting differences are identical to measurement noise.
And it is also the case that $W$ can be zero or have as many zero
eigenvalues as you like.
Yet another comment is that the expectation $\mu$ is an inverse-variance-weighted
mean of zero and the spectral difference $[y_1-y_2]$. And so on!

Now our suggestion (above) was that the scalar we would use to decide
whether two vectors are identical is the posterior pdf $p(\Delta\given
y_1,y_2)$ for the difference $\Delta$, evaluated at $\Delta=0$, or
perhaps the log of that:
\begin{eqnarray}
  Q &\equiv& -2\,\log p(\Delta=0\given y_1,y_2)
\\
    &=& \tra{\mu}\cdot\inv{\Sigma}\cdot\mu
\\
    &=& \tra{[y_1-y_2]}\cdot\inv{[V + C_1 + C_2 + W]}\cdot V\cdot\inv{[C_1 + C_2 + W]}\cdot[y_1-y_2]
\label{eq:scalar}\quad ,
\end{eqnarray}
where we have dropped some constants and used some more matrix foo.

This scalar $Q$ has lots of good properties, and lots to be discussed.
First, it is probabilistically justified, in the context of the explicit
assumptions I listed in the Assumptions section, above. That beats most
heuristics.
Second, it reduces to exactly $\chi^2$ of \equationname~(\ref{eq:chisquared}) in
precisely the limit where we would expect it. This limit is that in which the
intrinsic variance $W$ vanishes (so the noise is just $C_1 + C_2$), and the prior
variance is large in all directions (so $V$ dominates the other matrices in its
sum). That is, this whole formalism strongly endorses my original advice to ``just''
use $\chi^2$.
Third, if $V$ is low-rank (so there is no prior variance along some directions),
we only care about the data directions that have non-zero prior variance. The
multiply by $V$ in the middle of \equationname~(\ref{eq:scalar}) projects the
data difference down to the non-zero directions in data space, as we want.
Fourth, while the scalar $Q$ reduces to $\chi^2$ in the trivial limit,
it also obeys another intuitive limit, which is that if the prior
variance tensor $V$ and the observational noise $C_1 + C_2$ are at
odds, the projection onto one and then the other ought to appear. This
is like looking at the noise in the space in which the prior variance
is spherical, or looking at the prior information in the space in
which the observational information is isotropic.\footnote{This aspect
  might project onto things that Andy Gould (OSU) was saying at people
  in Heidelberg.} Or whatevs.

So---in sum---we weren't so wrong to be using $\chi^2$ in the first
place.
The biggest issue with our scalar $Q$ is that---as with $\chi^2$---we have to choose a
threshold. Choosing that threshold, if we want to (say) make a sample
of twins, is necessary, and it is also necessarily heuristic.

\paragraph{Linear algebra:}
The patient reader may complain that there has already been enough
linear algebra.\footnote{Even though there hasn't been nearly as much
  here as there has been in my notebooks getting to this point!}
But there hasn't, because I have been talking about the point that
$V$ might be low rank, and how this might help us, or how we might modify
the way we compute things if it \emph{is} low-rank.

HOGG: TBD: STUFF ABOUT V GETTING LOW RANK.

\paragraph{Stellar spectra and abundances:}
HOGG: TBD: SAY THINGS ABOUT THE SPECIFIC CASES OF Ness AND Hawkins.

\end{document}
